%-----------------------------------------------------------------------------------------------------------------------------
% \chapter{Machine Learning-Based Intrusion Detection System Approaches}
% This section will provide an overview of various approaches that have been developed for intrusion detection.
% Many works are being carried out in this context to find the best parameters and results for intrusion detection in various environments, and this survey will examine some of the most recent studies. The section will conclude with a summary table of the reviewed approaches.\\
% ============================== ARTICLE 1 =====================================
\section{Towards Realizing the Vision of Precision Medicine: AI-Based Prediction of Clinical Drug Response}
\textbf{Article Reference:} \cite{article_1}

\subsection*{Overview}
This study uses machine learning to predict patient response to the epilepsy drug brivaracetam using integrated clinical and genomic data. The resulting model demonstrated strong performance (AUC: 0.76 training, 0.75 validation) and identified specific biomarkers associated with poor response. The research underscores the potential of ML models to support precision medicine and optimize clinical trials by targeting likely responders. This study highlights the potential of AI to personalize treatment strategies in epilepsy by predicting drug response, a key aspect of personalized medicine.

\subsection*{Dataset}
\begin{itemize}
    \item \textbf{Discovery dataset:} 235 adult patients from a phase III clinical trial (NCT01261325).
    \item \textbf{External validation dataset:} 47 patients from an independent trial (NCT00490035).
\end{itemize}

\subsection*{Processing}
Clinical data included demographic and seizure-related information. Whole Genome Sequencing (WGS) data ($\sim$20 million variants) was filtered down to 40 features through knowledge-driven extraction, focusing on epilepsy-related genes and drug mechanism (e.g., SV2A gene, eQTLs). Genetic features included mutational load scores, polygenic risk scores, and structural variant descriptors.

\subsection*{ML Approach}
Multiple ML models were evaluated: sparse multi-block PLS-DA, multimodal neural networks, elastic net, gradient-boosted decision trees (GBDT), and stacked classifiers. The best performance was achieved using a GBDT model integrating all data types. GBDT models are well-suited for handling the complex interactions between clinical and genetic features, which is crucial for personalized drug response prediction. However, the inherent complexity of GBDT models can make it challenging to interpret the specific contributions of individual features, a limitation that future explainable AI (XAI) techniques could address.

\subsection*{Results}
The GBDT classifier achieved:
\begin{itemize}
    \item AUC (training): 0.76
    \item AUC (validation): 0.75
\end{itemize}

\subsection*{Future Directions and Challenges}
\begin{itemize}
    \item Addressing high dimensionality and sparsity of genomic data. This is a common challenge in personalized medicine research, as genomic data often has many variables but few samples.
    \item Integrating additional data types (e.g., EEG, imaging) to improve model performance. Multimodal data integration is essential for a holistic view of the patient but increases complexity.
    \item Generalizing models to other anti-epileptic drugs. This is crucial for wider clinical applicability in personalized epilepsy treatment.
    \item Collaborating with regulatory bodies for clinical adoption. AI-driven personalized medicine tools require rigorous validation and regulatory approval for safe and effective use.
    \item Increasing dataset size to enhance model performance (targeting $\sim$350 patients for AUC = 0.9). Larger datasets are vital for building robust and generalizable predictive models in personalized healthcare.
\end{itemize}

\subsection*{Critique}
\begin{itemize}
    \item The sample size, while sufficient for the study, could be larger to further enhance model performance and generalizability. 
    \item The complexity of the GBDT model, while providing good predictive power, makes it difficult to interpret the specific contributions of individual features.
    
\end{itemize}

% ============================== ARTICLE 2 =====================================
\section{Diabetes Prediction Using Machine Learning and Explainable AI Techniques}
\textbf{Article Reference:} \cite{article_2}

\subsection*{Overview}
This study proposes an automated diabetes prediction system using ML and explainable AI. The system combines the public Pima Indian dataset with a private dataset collected from female workers in a Bangladeshi textile factory. The system addresses data imbalance, missing values, and is deployed for real-time prediction via web and mobile applications. The development of non-invasive AI-driven tools for diabetes detection, as presented in this paper, contributes to personalized healthcare by enabling earlier and more accessible diagnosis.

\subsection*{Dataset}
\begin{itemize}
    \item \textbf{Pima Indian Dataset:} 768 records, 268 diabetes-positive; includes 8 features.
    \item \textbf{RTML Private Dataset:} 203 female employees; features similar to Pima dataset but lacks insulin values.
\end{itemize}

\subsection*{Processing}
\begin{itemize}
    \item Zero values in the merged dataset were replaced with corresponding mean values and the dataset was separated into training and test sets using the holdout validation technique.
    \item Mutual information was used to measure the interdependence of variables and feature importance.
    \item A semi-supervised approach using the extreme gradient boosting technique (XGB regressor) was used to predict the missing insulin feature of the RTML dataset.
\end{itemize}

\subsection*{ML Approach}
Various models were tested: decision trees, KNN, SVM, random forest, logistic regression, AdaBoost, XGBoost, bagging, and voting classifiers. Hyperparameters were tuned using GridSearchCV. The final model employed XGBoost with ADASYN for balancing. The choice of XGBoost is appropriate due to its effectiveness in handling complex datasets, but the lack of inherent explainability highlights the need for methods.

\subsection*{Results}
The best results were obtained using the XGBoost classifier with ADASYN:
\begin{itemize}
    \item Accuracy: 81\%
    \item F1 Score: 0.81
    \item AUC: 0.84
\end{itemize}

\subsection*{Challenges}
\begin{itemize}
    \item Missing insulin values required imputation via semi-supervised learning. This introduces a degree of uncertainty into the model.
    \item Class imbalance necessitated oversampling (SMOTE, ADASYN). Oversampling techniques can sometimes lead to overfitting.
    \item Limited private dataset size may hinder generalizability. Larger, more diverse datasets would improve the robustness of the model.
\end{itemize}

\subsection*{Future Directions}
\begin{itemize}
    \item Expanding dataset size for better robustness.
    \item Integrating fuzzy logic and optimization for improved prediction.
\end{itemize}
\subsection*{Critique}
\begin{itemize}
    \item The use of imputation for missing insulin values introduces some uncertainty.    
    \item The private dataset is relatively small, which may limit the model's generalizability.    
    
\end{itemize}

% ============================== ARTICLE 3 =====================================
\section{Integrating Machine Learning and Deep Learning Techniques for Advanced Alzheimer’s Disease Detection through Gait Analysis}
\textbf{Article Reference:} \cite{article_3}

\subsection*{Overview}
The paper aims to enhance early detection of Alzheimer’s Disease (AD) by leveraging gait analysis combined with advanced machine learning (ML) and deep learning (DL) techniques. Gait abnormalities, such as reduced stride length and irregular cadence, are identified as early biomarkers for cognitive decline associated with AD. The study emphasizes the need for non-invasive, scalable diagnostic tools. This research highlights the potential of AI-driven gait analysis to contribute to personalized AD management through early detection.

\subsection*{Dataset}
Data were collected using wearable sensors and motion capture systems in both clinical and real-world environments, providing high-resolution temporal and spatial gait metrics. The dataset includes gait features like stride length, cadence, swing time, and gait variability, with some data sourced from publicly available repositories like the UCI Machine Learning Repository. Preprocessing steps involved normalization, handling missing data via median imputation, class balancing with SMOTE, and feature selection through Recursive Feature Elimination.
\subsection*{Processing}
\begin{itemize}
    \item Normalization: Features were scaled between 0 and 1 to standardize the data, ensuring that features with larger ranges (e.g., stride length) did not dominate the model training.
    \item Handling Missing Data: Missing values were imputed using median substitution to maintain data integrity and reduce bias.
    \item Class Imbalance: The Synthetic Minority Over-sampling Technique (SMOTE) was applied to generate synthetic samples of the minority class (AD patients), addressing class imbalance issues.
    \item Feature Selection: Recursive Feature Elimination (RFE) was used to identify the most significant gait features—such as stride length, gait variability, and cadence—to improve model performance.
    \item Correlation Analysis: High correlations between key features (e.g., stride length and step length) validated their importance for prediction, informing feature selection.
\end{itemize}

\subsection*{ML Approach}
The study employed a hybrid deep learning model comprising Convolutional Neural Networks (CNNs) and Recurrent Neural Networks (RNNs) to classify individuals as healthy or at risk for AD. These models analyzed temporal-spatial gait features, capturing sequential patterns and irregularities. Traditional ML classifiers such as Random Forest and SVM were also evaluated for comparison. The use of a hybrid CNN-RNN model is a strength, as it leverages the capabilities of both CNNs for spatial feature extraction and RNNs for temporal sequence modeling, which is well-suited for gait analysis.

\subsection*{Results}
The hybrid CNN-RNN model achieved the highest accuracy of 93\%, with other metrics like precision, recall, and F1-score also indicating strong performance. Traditional models like Random Forest and SVM performed well but with slightly lower accuracy (88\% and 86\%, respectively). These results demonstrate the potential of deep learning models in accurately detecting early AD.

\subsection*{Challenges}
\begin{itemize}
    \item The reliance on controlled datasets, which may not fully reflect real-world variability, impacting model robustness.
    \item The complexity and interpretability of deep learning models, posing a barrier for clinical acceptance.
    \item The need for large, diverse datasets to ensure generalizability.
    \item Integration into clinical workflows and validation through real-world testing.
\end{itemize}

\subsection*{Future Directions}
\begin{itemize}
    \item Incorporating multimodal data sources, such as MRI, PET scans, vocal, and cognitive measures, to improve diagnostic precision.
    \item Expanding datasets to include diverse populations and environmental conditions, enhancing model robustness.
    \item Developing explainable AI frameworks to improve interpretability and clinician trust.
    \item Extending studies to include longitudinal gait data for monitoring disease progression and enabling earlier detection.
    \item Conducting clinical pilot studies and developing affordable wearable technologies for widespread, low-resource application.
\end{itemize}
\subsection*{Critique}
\begin{itemize}
    \item The dataset may not fully represent the variability of real-world gait patterns.    
    \item Deep learning models are often considered "black boxes," which can hinder clinical acceptance.
\end{itemize}

% ============================== ARTICLE 4 =====================================
\section {Diabetes detection using deep learning algorithms}
\textbf{Article Reference:} \cite{article_4}

\subsection*{Overview}
The authors developed a non-invasive method to detect diabetes using heart rate variability (HRV) signals derived from ECG data. They designed a deep learning architecture combining convolutional neural networks (CNN) and long short-term memory (LSTM) networks to automatically extract complex features from the HRV signals. These features were then classified using a support vector machine (SVM) with an RBF kernel. The approach achieved a high accuracy of 95.7\%, outperforming previous methods. The dataset consisted of ECG recordings from 20 individuals, each providing 10-minute samples, which were processed to extract HRV data without extensive preprocessing. The study demonstrated that deep learning models could effectively identify diabetes from HRV signals, with future work aimed at expanding datasets, improving model robustness, and exploring anomaly prediction for earlier diagnosis. This research demonstrates the potential of AI for non-invasive, personalized diabetes screening.
\subsection*{Dataset}
\begin{itemize}
    \item ECG recordings from 20 individuals (both diabetic and normal).
    \item Each participant provided a 10-minute ECG sample, from which heart rate time series data was derived.
    \item Total datasets: 71 datasets for both groups, each containing 1000 samples.
\end{itemize}

\subsection*{Processing}
\begin{itemize}
    \item Used Pan and Tompkins algorithm for QRS complex detection to extract heart rate intervals.
    \item Derived HRV signals directly from ECG without additional preprocessing.
    \item Input data fed into deep learning architectures for automatic feature learning.
\end{itemize}

\subsection*{ML Approach}
\begin{itemize}
    \item Built a deep learning model comprising 5 CNN layers followed by an LSTM layer to capture spatial and temporal features.
    \item Used dropout (0.1) for regularization.
    \item Extracted features automatically within the network, then classified using an SVM with RBF kernel.
    \item Employed 5-fold cross-validation for robust evaluation. The combination of CNNs and LSTMs is well-suited for processing time-series data like HRV signals.
\end{itemize}
\subsection*{Results}
The study achieved a maximum classification accuracy of 95.7\% using a CNN-LSTM architecture combined with SVM for placement of the final classifier. This result represents the highest accuracy reported so far for non-invasive diabetes detection using HRV signals as input. The detailed results showed that:

\begin{itemize}
    \item The combination of deep learning feature extraction with SVM classification outperformed using deep learning alone.
    \item Various architectures tested yielded accuracies ranging from around 68.\% (CNN 1 with SVM) to 95.7\% (CNN 5-LSTM with SVM).
    \item The high accuracy indicates that the proposed model effectively captures the complex temporal and spatial features of HRV signals associated with diabetic and normal individuals, confirming the potential of this approach for reliable, non-invasive diabetes detection.
\end{itemize}
\subsection*{Challenges}
\begin{itemize}
    \item Limited dataset size could affect generalization; larger datasets are needed.
    \item Variability in HRV signals due to individual differences may pose challenges.
    \item Ensuring model interpretability for clinical acceptance.
    \item Moving from controlled datasets to real-world, noisy ECG signals.
\end{itemize}

\subsection*{Future Directions}
\begin{itemize}
    \item Increase dataset size to improve model accuracy and robustness.
    \item Explore anomaly prediction techniques by analyzing dynamic characteristics in HRV data.
    \item Develop more advanced deep learning models for early and accurate detection.
    \item Investigate applicability to real-time monitoring and broader clinical validation.
\end{itemize}
\subsection*{Critique}
\begin{itemize}
    \item The dataset size is limited, which may affect the model's ability to generalize to larger populations.
    \item Like other deep learning models, the interpretability of the model could be a concern for clinical use.
\end{itemize}

\section{Comparison of the Solutions}
The table below compares the reviewed studies based on disease domain, dataset, preprocessing methods, approach, and results.

\begin{table}[htbp]
    \begin{adjustwidth}{-2cm}{-2cm}
    \centering
    \begin{tabular}{|p{1.5cm}|p{2.5cm}|p{3cm}|p{3.5cm}|p{2.5cm}|p{2.5cm}|}
    \hline
    \textbf{Work} & \textbf{Disease/Domain} & \textbf{Dataset} & \textbf{Data Processing} & \textbf{Approach} & \textbf{Results} \\
    \hline
    \cite{article_1} & Epilepsy & Phase III (235) + Validation (47) patients & Clinical + WGS feature extraction (e.g., SV2A), mutational scores, PRS & Gradient-Boosted Decision Trees & AUC: 0.76 (train), 0.75 (validation) \\
    \hline
    \cite{article_2} & Diabetes Prediction & Pima Indian (768) + RTML (203) records & Imputation, ADASYN, Mutual Info, Holdout Validation & XGBoost + Ensemble Methods (voting, bagging) & AUC: 0.84, Accuracy: 81\%, F1 Score: 0.81 \\
    \hline
    \cite{article_3} & Alzheimer’s Disease & Wearable sensors and motion capture data & Normalization, median imputation, SMOTE, RFE, correlation analysis & Hybrid CNN-RNN (LSTM) & Accuracy: 93\%, Precision: 92\%, Recall: 91\%, F1-Score: 91.5\%, AUC-ROC: 95\% \\
    \hline
    \cite{article_4} & Diabetes & ECG recordings (71 datasets) & Pan-Tompkins for QRS detection & CNN-LSTM + SVM & Accuracy: 95.7\% \\
    \hline
    \end{tabular}
    \caption{Comparison of AI Approaches in Health Applications}
    \label{tab:ai_health_comparison}
    \end{adjustwidth}
\end{table}

% \newpage
% \section{Discussion}
% As presented in the table above, different methods were proposed using machine learning which are seen as the most effective technique to build an intrusion detection system and detect attacks with different datasets mostly NSL-KDD dataset which resulted in a good accuracy.\\
% Nearly half of the studies did not work with normalization, there are also studies that did not perform a feature selection, as well as the most did not test many learning algorithms and the most of studies work on binary classification (normal or attack).

 
% \section{Conclusion}
% In this chapter, we studied and evaluated the latest methods proposed by recent research, about the use of machine learning methods for attacks classification based on data-sets.

% Most of the studied architectures could more or less achieve satisfying results with high accuracy. However, further research using large and fresh data-sets is always required to improve the performance of the existing methods for more efficient results.



