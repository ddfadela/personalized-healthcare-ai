%-----------------------------------------------------------------------------------------------------------------------------
% \chapter{Machine Learning-Based Intrusion Detection System Approaches}
% This section will provide an overview of various approaches that have been developed for intrusion detection.
% Many works are being carried out in this context to find the best parameters and results for intrusion detection in various environments, and this survey will examine some of the most recent studies. The section will conclude with a summary table of the reviewed approaches.\\
\section{Towards Realizing the Vision of Precision Medicine: AI-Based Prediction of Clinical Drug Response}
\textbf{Article Reference:} \cite{article_1}

\subsection*{Overview}
This study uses machine learning to predict patient response to the epilepsy drug brivaracetam using integrated clinical and genomic data. The resulting model demonstrated strong performance (AUC: 0.76 training, 0.75 validation) and identified specific biomarkers associated with poor response. The research underscores the potential of ML models to support precision medicine and optimize clinical trials by targeting likely responders.

\subsection*{Dataset}
\begin{itemize}
    \item \textbf{Discovery dataset:} 235 adult patients from a phase III clinical trial (NCT01261325).
    \item \textbf{External validation dataset:} 47 patients from an independent trial (NCT00490035).
\end{itemize}

\subsection*{Processing}
Clinical data included demographic and seizure-related information. Whole Genome Sequencing (WGS) data (~20 million variants) was filtered down to 40 features through knowledge-driven extraction, focusing on epilepsy-related genes and drug mechanism (e.g., SV2A gene, eQTLs). Genetic features included mutational load scores, polygenic risk scores, and structural variant descriptors.

\subsection*{Model Building}
Multiple ML models were evaluated: sparse multi-block PLS-DA, multimodal neural networks, elastic net, gradient-boosted decision trees (GBDT), and stacked classifiers. The best performance was achieved using a GBDT model integrating all data types.

\subsection*{Results}
The GBDT classifier achieved: AUC (training): 0.76

\subsection*{Future Directions and Challenges}
\begin{itemize}
    \item Addressing high dimensionality and sparsity of genomic data.
    \item Integrating additional data types (e.g., EEG, imaging) to improve model performance.
    \item Generalizing models to other anti-epileptic drugs.
    \item Collaborating with regulatory bodies for clinical adoption.
    \item Increasing dataset size to enhance model performance (targeting ~350 patients for AUC = 0.9).
\end{itemize}

\section{Diabetes Prediction Using Machine Learning and Explainable AI Techniques}
\textbf{Article Reference:} \cite{article_2}

\subsection*{Overview}
This study proposes an automated diabetes prediction system using ML and explainable AI. The system combines the public Pima Indian dataset with a private dataset collected from female workers in a Bangladeshi textile factory. The system addresses data imbalance, missing values, and is deployed for real-time prediction via web and mobile applications.

\subsection*{Dataset}
\begin{itemize}
    \item \textbf{Pima Indian Dataset:} 768 records, 268 diabetes-positive; includes 8 features.
    \item \textbf{RTML Private Dataset:} 203 female employees; features similar to Pima dataset but lacks insulin values.
\end{itemize}

\subsection*{Processing}
\begin{itemize}
    \item Zero values in the merged dataset were replaced with corresponding mean values and the dataset was separated into training and test sets using the holdout validation technique.
    \item Mutual information was used to measure the interdependence of variables and feature importance.
    \item A semi-supervised approach using the extreme gradient boosting technique (XGB regressor) was used to predict the missing insulin feature of the RTML dataset.
\end{itemize}

\subsection*{ML Approach}
Various models were tested: decision trees, KNN, SVM, random forest, logistic regression, AdaBoost, XGBoost, bagging, and voting classifiers. Hyperparameters were tuned using GridSearchCV. The final model employed XGBoost with ADASYN for balancing.

\subsection*{Results}
The best results were obtained using the XGBoost classifier with ADASYN:
\begin{itemize}
    \item Accuracy: 81\%
    \item F1 Score: 0.81
    \item AUC: 0.84
\end{itemize}

\subsection*{Challenges}
\begin{itemize}
    \item Missing insulin values required imputation via semi-supervised learning.
    \item Class imbalance necessitated oversampling (SMOTE, ADASYN).
    \item Limited private dataset size may hinder generalizability.
\end{itemize}

\subsection*{Future Directions}
\begin{itemize}
    \item Expanding dataset size for better robustness.
    \item Integrating fuzzy logic and optimization for improved prediction.
\end{itemize}

\section{Integrating Machine Learning and Deep Learning Techniques for Advanced Alzheimer’s
Disease Detection through Gait Analysis}
\textbf{Article Reference:} \cite{article_3}

\subsection*{Overview}
The paper aims to enhance early detection of Alzheimer’s Disease (AD) by leveraging gait analysis combined with advanced machine learning (ML) and deep learning (DL) techniques. Gait abnormalities, such as reduced stride length and irregular cadence, are identified as early biomarkers for cognitive decline associated with AD. The study emphasizes the need for non-invasive, scalable diagnostic tools.


\subsection*{Dataset}
Data were collected using wearable sensors and motion capture systems in both clinical and real-world environments, providing high-resolution temporal and spatial gait metrics. The dataset includes gait features like stride length, cadence, swing time, and gait variability, with some data sourced from publicly available repositories like the UCI Machine Learning Repository. Preprocessing steps involved normalization, handling missing data via median imputation, class balancing with SMOTE, and feature selection through Recursive Feature Elimination.
\subsection*{Processing}
- Normalization: Features were scaled between 0 and 1 to standardize the data, ensuring that features with larger ranges (e.g., stride length) did not dominate the model training.
- Handling Missing Data: Missing values were imputed using median substitution to maintain data integrity and reduce bias.
- Class Imbalance: The Synthetic Minority Over-sampling Technique (SMOTE) was applied to generate synthetic samples of the minority class (AD patients), addressing class imbalance issues.
- Feature Selection: Recursive Feature Elimination (RFE) was used to identify the most significant gait features—such as stride length, gait variability, and cadence—to improve model performance.
- Correlation Analysis: High correlations between key features (e.g., stride length and step length) validated their importance for prediction, informing feature selection.

\subsection*{ML Approach}
The study employed a hybrid deep learning model comprising Convolutional Neural Networks (CNNs) and Recurrent Neural Networks (RNNs) to classify individuals as healthy or at risk for AD. These models analyzed temporal-spatial gait features, capturing sequential patterns and irregularities. Traditional ML classifiers such as Random Forest and SVM were also evaluated for comparison.
\subsection*{Results}
The hybrid CNN-RNN model achieved the highest accuracy of 93\%, with other metrics like precision, recall, and F1-score also indicating strong performance. Traditional models like Random Forest and SVM performed well but with slightly lower accuracy (88\% and 86\%, respectively). These results demonstrate the potential of deep learning models in accurately detecting early AD.

\subsection*{Challenges}
\begin{itemize}
    \item The reliance on controlled datasets, which may not fully reflect real-world variability, impacting model robustness.
    \item The complexity and interpretability of deep learning models, posing a barrier for clinical acceptance.
    \item The need for large, diverse datasets to ensure generalizability.
    \item Integration into clinical workflows and validation through real-world testing.
\end{itemize}

\subsection*{Future Directions}
\begin{itemize}
    \item Incorporating multimodal data sources, such as MRI, PET scans, vocal, and cognitive measures, to improve diagnostic precision.
    \item Expanding datasets to include diverse populations and environmental conditions, enhancing model robustness.
    \item Developing explainable AI frameworks to improve interpretability and clinician trust.
    \item Extending studies to include longitudinal gait data for monitoring disease progression and enabling earlier detection.
    \item Conducting clinical pilot studies and developing affordable wearable technologies for widespread, low-resource application.
\end{itemize}


\newpage
\section{Comparison of the Solutions}

The table below compares the reviewed studies based on disease domain, dataset, preprocessing methods, approach, and results.
\begin{table}[htbp]
    \begin{adjustwidth}{-2cm}{-2cm} 
    \centering 
    \begin{tabular}{|p{1.5cm}|p{2.5cm}|p{3cm}|p{3.5cm}|p{2.5cm}|p{2.5cm}|}
    \hline
    \textbf{Work} & \textbf{Disease/Domain} & \textbf{Dataset} & \textbf{Data Processing} & \textbf{Approach} & \textbf{Results} \\
    \hline
    \cite{article_1} & Epilepsy & Phase III (235) + Validation (47) patients & Clinical + WGS feature extraction (e.g., SV2A), mutational scores, PRS & Gradient-Boosted Decision Trees & AUC: 0.76 (train), 0.75 (validation) \\
    \hline
    \cite{article_2} & Diabetes Prediction & Pima Indian (768) + RTML (203) records & Imputation, ADASYN, Mutual Info, Holdout Validation & XGBoost + Ensemble Methods (voting, bagging) & AUC: 0.84, Accuracy: 81\%, F1 Score: 0.81 \\
    \hline
    \cite{article_3} & Alzheimer’s Disease & 300 instances from wearable sensors and motion capture & Normalization, median imputation, SMOTE, RFE, correlation analysis & Hybrid CNN-RNN (LSTM) & Accuracy: 93\%, Precision: 92\%, Recall: 91\%, F1-Score: 91.5\%, AUC-ROC: 95\% \\
    \hline
    \end{tabular}
    \caption{Comparison of AI Approaches in Health Applications}
    \label{tab:ai_health_comparison}
    \end{adjustwidth}
\end{table}

% \newpage
% \section{Discussion}
% As presented in the table above, different methods were proposed using machine learning which are seen as the most effective technique to build an intrusion detection system and detect attacks with different datasets mostly NSL-KDD dataset which resulted in a good accuracy.\\
% Nearly half of the studies did not work with normalization, there are also studies that did not perform a feature selection, as well as the most did not test many learning algorithms and the most of studies work on binary classification (normal or attack).

 
% \section{Conclusion}
% In this chapter, we studied and evaluated the latest methods proposed by recent research, about the use of machine learning methods for attacks classification based on data-sets.

% Most of the studied architectures could more or less achieve satisfying results with high accuracy. However, further research using large and fresh data-sets is always required to improve the performance of the existing methods for more efficient results.



